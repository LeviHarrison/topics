\documentclass{article}
\usepackage[a4paper, total={7in, 9.5in}]{geometry}
\usepackage{enumitem}
\usepackage{amsthm}
\usepackage[makeroom]{cancel}
\usepackage{graphicx}
\graphicspath{ {./images/} }
\usepackage{amsfonts}
\usepackage[group-separator={,}]{siunitx}
\usepackage{multirow}
\usepackage{amsmath}

\begin{document}

\begin{center}
    \LARGE{Problem Set 10}\\[0.5em]
    \large{Levi Harrison}\\[0.5em]
    \small{February 12, 2024}
\end{center}

\,

\begin{enumerate}
    \item After The Seniors Leave

          Some years, after the seniors leave, there are only two juniors in Topics or Multi. Suppose we were in such a situation this year (say Levi and Raj just quit school, leaving only Daniel and Nathan, for example - we don't know who will quit, so we will call them Player 1 and Player 2). Mr. Bettendorf wants to give only one A in the fourth marking period, and he will give that A to the boy who puts in the most time on his final project. If they put in the same amount of time, Mr. B will flip a coin for the A. In terms of payoffs, the A is worth 3, and the B (or whatever) is 0; no time is 0, a little time is -1, and lot of time is -2.

          \begin{enumerate}[label=(\alph*)]
              \item Write down the game in matrix form.
          \end{enumerate}

          \,

          \centerline{Player 2}
          \begin{center}
              Player 1
              \begin{tabular}{|c|c|c|c|}
                  \hline
                                & No time  & A little time & A lot of time \\ \hline
                  No time       & 1.5, 1.5 & 0, 2          & 0, 1          \\ \hline
                  A little time & 2, 0     & .5, .5        & -1, 1         \\ \hline
                  A lot of time & 1, 0     & 1, -1         & -.5, -.5      \\ \hline
              \end{tabular}
          \end{center}

          \,

          \begin{enumerate}[resume]
              \item Are any pure strategies weakly or strictly dominated?
          \end{enumerate}

          Yes.

          \begin{enumerate}[resume]
              \item Is there a pure-strategy Nash equilibrium?
          \end{enumerate}

          No.

          \begin{enumerate}[resume]
              \item Is there a mixed-strategy Nash equilibrium?
          \end{enumerate}

          We can prove this using the Nash Existance Theorem.

          Let's consider the payoff functions for Player 1 ($v_1(N, ...), v_1(L, ...), v_1(M, ...)$). In order for there to be a mixed-strategy Nash equilibrium, all 3 should be equivalent. Let's call the probabilites of Player 2 choosing strategies N (no time), L (a little time), and M (a lot of time) $p$, $q$, and $1 - p - q$ respectively.

          Thus, we have that:

          \[v_1(N, p, q) = v_1(L, p, q) = v_1(M, p, q)\]
          \[1.5 \cdot p + 0 \cdot q + 0 \cdot (1 - p - q) = 2 \cdot p + .5 \cdot q - 1 \cdot (1 - p - q) = 1 \cdot p + 1 \cdot q - .5 \cdot (1 - p - q)\]
          \[1.5p = 2p + .5q - 1 + p + q = p + q - .5 + .5p + .5q\]
          \[1.5p = 3p + 1.5q - 1 = 1.5p + 1.5q - .5\]

          \[1.5p = 3p + 1.5q - 1\]
          \[1.5p = 1 - 1.5q\]
          \[p = \frac{2}{3} - q\]

          \[3p + 1.5q - 1 = 1.5p + 1.5q - .5\]
          \[1.5p = .5\]
          \[p = \frac{1}{3}\]
          \[\frac{2}{3} - q = \frac{1}{3}\]
          \[q = \frac{1}{3}\]

          \[1.5p = 3p + 1.5q - 1\]
          \[1 = 1.5p + 1.5q\]
          \[1 = 1.5p + 1.5 \cdot (\frac{1}{3})\]
          \[1 = 1.5p + .5\]
          \[.5 = 1.5p\]
          \[p = \frac{1}{3}\]

          \[1 - p - q = 1 - \frac{1}{3} = \frac{1}{3} = 1/3\]

          The same proportions ($\frac{1}{3}, \frac{1}{3}, \frac{1}{3}$) apply to player 2 as well since the game is symetrical, so there exists a mix-strategy Nash equilibrium choosing each strategy with equal probability.

    \item Cops and Robbers

          Cooper is a cop, and Ralph is a robber. Cooper values hanging out at Dunkin quite a bit; let's call that experience a 10. If he were to go on patrol, he might catch Ralph in a robbery, which is worth 20. If he doesn't catch Ralph, that's 0. Ralph, on the other hand, could always hide, but that gets him nothing. If he goes out for a robbery and Cooper is having coffee, get scores a heist (+10); if, on the other hand, Cooper is *not* hanging out in donut heaven, Ralph will surely be caught (-10).

          \begin{enumerate}[label=(\alph*)]
              \item Write down the game in matrix form.
          \end{enumerate}

          \centerline{Ralph}
          \begin{center}
              Cooper
              \begin{tabular}{|c|c|c|}
                  \hline
                                   & Hide  & Rob     \\ \hline
                  Stay at Dunkin   & 10, 0 & 10, 10  \\ \hline
                  Go out on patrol & 0, 0  & 20, -10 \\ \hline
              \end{tabular}
          \end{center}

          \begin{enumerate}[resume]
              \item Is there a pure-strategy Nash equilibrium?
          \end{enumerate}

          No.

          \begin{enumerate}[resume]
              \item Is there a mixed-strategy Nash equilibrium?
          \end{enumerate}

          Once again, we can prove this using the Nash existance theorem

          We'll start with the payoff functions for player 1, Cooper, and assign his two possible actions, stay at Dunkin and patrol, D and P respectively. We'll call the probabilites of player 2, Ralph choosing his actions (hiding and robbing) p and 1 - p.

          \[v_1(D, p) = v_1(P, p)\]
          \[10 \cdot p + 10 \cdot (1 - p) = 0 \cdot p + 20 \cdot (1 - p)\]
          \[10p + 10 - 10p = 20 - 20p\]
          \[20p = 10\]
          \[p = .5\]

          \[1 - p = .5\]

          So, there exists a pure-strategy Nash equilibrium for Cooper when he chooses each strategy with equal probability. Now let's do the same for Ralph, using the actions H (hide) and R (rob), and the probabilites of Cooper choosing his respective actions of q and 1 - q.

          \[v_2(H, q) = v_2(R, q)\]
          \[0 \cdot q + 0 \cdot (1 - q) = 10 \cdot q - 10 \cdot (1 - q)\]
          \[0 = 10q - 10 + 10q\]
          \[20q = 10\]
          \[q = .5\]

          \[1 - q = .5\]

          So, the pure-strategy Nash equilibrium for Ralph also exists with him choosing each strategy with equal probability.

    \item Continuous All Pay Auction

          Consider an all-pay auction for a one-million-dollar bill. Let's count in millions and say $S_i \in [0, 1]$. Players care only about the expected value they will end up with (so if I bet 0.4 and expect to win with probability 0.7 then I get 0.3).

          \begin{enumerate}[label=(\alph*)]
              \item Write out the model as a normal-form game. (This is basically in your notes already.)
          \end{enumerate}

          \[N = \{1, 2\}\]
          \[S_i \in [0, 1]\]
          \[v_i(s_i, s_j) = \begin{cases}
                  1 - s_i           & s_i > s_j \\
                  \frac{1}{2} - s_i & s_i = s_j \\
                  -s_i              & s_i < s_j
              \end{cases}
          \]

          \begin{enumerate}[resume]
              \item Show that this game has no pure-strategy Nash equilibrium by considering cases such as $s_i = s_j = 1$, $s_i = s_j < 1$, and $s_i > s_j \geq 0$; in the last case, for example, what might player i do to increase his payoff? (It can be useful to express tiny amounts by the traditional letter $\epsilon > 0$).
          \end{enumerate}

          If $s_i = s_j = 1$, the best response for player 1 will be to just move their bet to 0, because player 2 is betting the full million dollars, so if player 1 were to match that he would end up with a negative payoff. If player 1 goes lower, he will certainly loose the action anyway, so he should just bet 0.

          If $s_i = s_j < 1$, $s_i$ can just move his bet $\epsilon$ higher (where $\epsilon \in (0, \frac{1 - s_i}{2})$), which will win him the auction with an expected value of $1 - s_i - \epsilon > 1 - [\frac{1 - s_i}{2}] = \frac{1}{2} - \frac{s_i}{2} > \frac{1}{2} - s_i$.

          If $s_i > s_j \geq 0$, player 1 can just move his bet $\epsilon$ lower, where $\epsilon \in (s_i - s_j, s_i)$, so then he will still win the auction but spend less to do it (payoff of $1 - s_i + \epsilon > 1 - s_i$).

          Therefore, for all possible strategy profiles there is a better response for player 1 (and therefore also player 2), so there is no pure strategy Nash equilibrium.

          \begin{enumerate}[resume]
              \item If each player i chooses an interval $[\underline{x_i}, \overline{x_i}]$ with $0 \leq x_i < x_i \leq 1$ via a continuous probability distribution over the interval, show that any mixed-strategy Nash equilibrium would require that $\underline{x_1} = \underline{x_2} = 0$ and $\overline{x_1} = \overline{x_2} = 1$. Use those facts to argue that if two such strategies are a Nash equilibrium then both players must be getting an expected payoff of zero.
          \end{enumerate}

          Lower bounds have to be equal cause cool graph thing ($x_1 = x_2$)

          No player will choose lower bound value cause they will loose with probability 1 cause the other player will choose that with probability 0, so will be negative playoff. Because probability that both choose it is 0, so one will always loose.

          So player will try to minimize loss and choose 0 because then he won't loose anything, avoiding negative payoff.

          Player 1 will choose upper bound because he will always win, so player 2 will also choose upper bound to match that, and move up to win.

          By def, because it's a Nash equilibrium anything in the support must have the same expected value, and because 0 is in the support then the expected payoff is zero.
\end{enumerate}

\end{document}